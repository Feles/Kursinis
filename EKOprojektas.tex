\documentclass[a4paper]{article}

\usepackage[utf8]{inputenc}
\usepackage[L7x]{fontenc}
\usepackage[lithuanian]{babel}
\usepackage{lmodern}

\title{ Kursinis darbas "Lietuvos infliacijos modelis"}
\date{2011 rugsėjo 21 diena}
\author{Daiva Kemzūraitė, Alma Mežanec, Laurynas Venčkauskas}
\begin{document}
\maketitle
\title\textbf{\LARGE{Planas:}}
\textit{}
\textit{}
\section{Svarba:}
\textnormal{  identifikavus Lietuvos ekonomikoje taikytiną infliacijos  modelį, galima tiksliau prognozuoti tolimesnę ekonomikos raidą bei siūlyti naudingesnius infliacijos kontrolės veiksmus.}
\section{Tyrimo objektas:}
\textnormal{  infliacija ir jos svyravimų priežasčių analizė Lietuvos ekonomikoje.}
\section{Tikslai:}

\begin{itemize}
\item  išsiaiškinti, nuo ko ir kaip priklauso infliacija; infliacijos priklausomybės tyrimas nuo konkrečių Lietuvos ekonomikos makroekonominių kintamųjų (infliacijos kontrolės priemonių).
\item Lietuvos ekonomikos infliacijos modelis.
\item infliacijos prognozė (prieš tai atrinkus modelį, kuris tiksliausiai paaiškina infliacijos svyravimus).
\item ekonominė apžvalga - kiekvieno rezultato projekcija į ekonomiką.
\end{itemize}

\section{Darbo struktūra;}
\subsection{Modelių, paaiškinančių infliacijos svyravimus užsienio šalių valstybėse,  paieška;}
\subsubsection{Modelių analizė, tyrinėjimas: kiekvieno iš jų svarba, norint ištirti ekonominių svyravimų įtaką Lietuvos infliacijai;}
\subsubsection{Modelio pasirinkimas pagal Lietuvos ekonomikos vyksmų paaiškinamumą (įtaką) infliacijai bei kintamųjų faktinių duomenų prieinamumą;}
\subsection{	Duomenų, reikalingų pasirinkto modelio realizacijai, paieška;}
\subsubsection{Duomenų analizė:}
\begin{itemize}
\item  duomenų apžvalga.
\item grafikai.
\end{itemize}
\subsubsection{Kiekvienos modelio komponentės stacionarumo tikrinimas;}
\subsection{Regresinių modelių kūrimas;}
\subsubsection{Ištirti Lietuvos ekonomikos infliacijos priklausomybę nuo modelyje naudojamų kintamųjų, sudarant tiesinę regresiją be ankstinių (toliau minimą kaip (1r));}
\begin{itemize}
\item Aptarti gautus rezultatus (koeficientų reikšmingumą, jų įverčius ir kitus regresinio modelio parametrus).
\item (1r) regresijos įvertintų reikšmių ir faktinių infliacijos reikšmių grafikų brėžimas.
\end{itemize}
\subsubsection{Ištirti Lietuvos ekonomikos infliacijos priklausomybę nuo modelyje naudojamų kintamųjų, sudarant vektorinę regresiją neįtraukiant ankstinių (toliau minima kaip (2r));}
\begin{itemize}
\item Aptarti gautus rezultatus (koeficientų reikšmingumą, jų įverčius ir kitus regresinio modelio parametrus).
\item (2r) regresijos įvertintų reikšmių ir faktinių infliacijos reikšmių grafikų brėžimas.
\item  (1r) ir (2r) infliacijos regresinių modelių palyginimas.
\end{itemize}
\subsubsection{Ištirti Lietuvos ekonomikos infliacijos priklausomybę nuo modelyje naudojamų kintamųjų, sudarant tiesinę regresiją ir  įtraukiant į ją ankstinius (toliau minima kaip (3r));}
\begin{itemize}
\item Padirbėti su modeliu, išmetant iš modelio nereikšmingus narius (jeigu tokių yra), paliekant modelį, kurio visi koeficientai yra reikšmingi, t.y. papeda nuspėti infliaciją, ją paaiškina.
\item Aptarti gautus rezultatus (koeficientų reikšmingumą, jų įverčius ir kitus regresinio modelio parametrus).
\item (3r) regresijos įvertintų reikšmių ir faktinių infliacijos reikšmių grafikų brėžimas.
\item (1r) , (2r) ir (3r) infliacijos regresinių modelių palyginimas.
\end{itemize}
\subsubsection{Ištirti Lietuvos ekonomikos infliacijos priklausomybę nuo modelyje naudojamų kintamųjų, sudarant vektorinę regresiją ir įtraukiant į ją ankstinius (toliau minima kaip (4r));}
\begin{itemize}
\item Palikti vektoriniame regresiniame  modelyje (4r)  tą ankstinių eilę, su kuria yra mažiausias AIC arba BIC kriterijus (pasirinktinai pagal situaciją).
\item Aptarti gautus rezultatus (koeficientų reikšmingumą, jų įverčius ir kitus regresinio modelio parametrus).
\item (4r) regresijos įvertintų reikšmių ir faktinių infliacijos reikšmių grafikų brėžimas.
\item (1r) , (2r) , (3r) bei (4r) infliacijos regresinių modelių palyginimas (pagal faktines infliacijos reikšmes).
\end{itemize}
\subsection{Remiantis atliktais keturiais regresiniais skaičiavimais, apibendrinti infliacijos priklausomumo  nuo visų kintamųjų ypatybes bei padaryti ekonomines išvadas;}
\subsection{	Infliacijos prognozė.}
\section{Literatūra (sąrašas nuolat pildomas):}
\subsection {www.stat.gov.lt;} 
\subsection{www.lb.lt;}
\subsection{O. Blanchard  „Makroekonomika“.}
\subsection{http://epp.eurostat.ec.europa.eu/}
\end{document}
