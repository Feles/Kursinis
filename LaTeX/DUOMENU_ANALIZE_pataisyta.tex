\documentclass[a4paper]{article}

\usepackage[utf8]{inputenc}
\usepackage[L7x]{fontenc}
\usepackage[lithuanian]{babel}
\usepackage{lmodern}
\usepackage{graphicx}
\usepackage{graphics}
\usepackage{float}





\title{ Kursinis darbas "Lietuvos infliacijos modelis"}
\date{2011 spalio 5 diena}
\author{\textbf{grupė:} Daiva Kemzūraitė, Alma Mežanec, Laurynas Venčkauskas; \\
\textbf{vadovas:} doc. dr. R. Lapinskas}


\begin{document}
\maketitle

%\section{Eonominė apžvalga}%
\title\textbf{\LARGE{Duomenų analizė}}

\textit{}
\textit{}
\section{Ekonominė apžvalga}
\textnormal{    \             \             Plačiausiai naudojamas infliacijos rodiklis yra vartotojų kainų indeksas.  Svyravimai infliacijos kurse yra interpretuojami kaip kainų lygio pokytis  iki nustatytos reikšmės, kitaip tariant, infliacija yra į pusiausvyrą grįžtantis procesas. Šis kainų dinamikos elgesys yra pagrindas šiuolaikinio požiūrio į infliacijos modeliavimą, kartu ir prognozei, ir elgsenos analizei.}


\section{Infliacijos modelis}
\textnormal{\         \ Bendrais bruožais apibūdinant pagrindinius kainų lygio modelio požymius, remsimės de Brouwer and Ericsson straipsnio " A MARKUP MODELOF INFLATION FORTHE EURO AREA"  (1998) aiškinimais.  Vis dėlto, mes šiek tiek pakoreguosime modelį, kad galėtume pritaikyti Lietuvos vartotojų kainų indeksui pagal mėnesines komponentes. }\\
\\
\textbf{  $VKI_t=\psi{MAL_t}^K{PM_t}^\beta{PNF_t}^\gamma{VIR_t}^\lambda{{e}^\varphi}^ {trend}$}
\begin{itemize}
\item $VKI$ - vartotojų kainų indeksas  (2006-2011 metai);
\item $\psi$ - laisvasis narys;
\item MAL - disponuojmos gyventojų pajamos Lietuvoje (2006-2011 metai); (šių duomenų dar ieškome)
\item PM - importo kainos (2006 - 2011 m.);
\item PNF - naftos kainos (2006 - 2011 m.);
\item VIR - suvidurkinta palūkanų norma (2006 - 2011 m.);
\item $trend$ - laikas.
\end{itemize}


\section{Laikinių sekų grafikai}

\begin{figure}[H]
\centering 
\includegraphics[width=6cm,  height=6cm]{VIK}
\caption{Vartotojų kainų indeksas}
\end{figure}

\begin{figure}[H]
\centering 
\includegraphics[width=6cm,  height=6cm]{1log}
\caption{Importo kainų indeksas}
\end{figure}


\begin{figure}[H]
\centering 
\includegraphics[width=6cm,  height=6cm]{naftan}
\caption{Naftos kainos}
\end{figure}

\begin{figure}[H]
\centering 
\includegraphics[width=6cm,  height=6cm]{PALNN}
\caption{Palūkanų norma dviejų metų indėliams}
\end{figure}
\newpage

\section{Logaritmuotų duomenų grafikai}
\begin{figure}[H]
\centering 
\includegraphics[width=6cm,  height=6cm]{VIKL}
\caption{Logaritmuotas vartotojų kainų indeksas}
\end{figure}

\begin{figure}[H]
\centering 
\includegraphics[width=6cm,  height=6cm]{graf1}
\caption{Logaritmuotas importo kainų indeksas}
\end{figure}



\begin{figure}[H]
\centering 
\includegraphics[width=6cm,  height=6cm]{nafl}
\caption{Logaritmuotos naftos kainos}
\end{figure}

\begin{figure}[H]
\centering 
\includegraphics[width=6cm,  height=6cm]{PALNL}
\caption{Logaritmuota palūkanų norma dviejų metų indėliams}
\end{figure}


\newpage
\section{Komponenčių stacionarumo tikrinimas}
\begin{itemize}
\item Vartotojų kainų indekso vienetinės šaknies testas:

\subsection{Vartotojų kainų indekso vienetinės šaknies testas.}
\textnormal {Kadangi  testo t statistika yra didesnė nei 5 procentų kritinė reikšmė $(-1.653 > -2.89)$, tai  procesas turi vienetinę šaknį}

\subsection {Importo kainų indekso vienetinės šaknies testas}
\textnormal{ Testo reikšme yra didesnė už 5 proc.  kritinę reikšmę $(-2.713>-3.45)$, todėl logVKI yra procesas su vienetine šaknimi (be trendo).}


\subsection{ Naftos kainos vienetinės šaknies testas:}
\textnormal{Kadangi testo t statistika nėra didesnė nei 5 proc. kritinė reikšmė $(-2.89 > -3.485)$,  procesas neturi vienetinės  šaknies.}


\subsection{Palūkanų normos vienetinės šaknies testas.}
\textnormal{Kadangi  testo t statistika yra didesnė nei 5 proc. kritinė reikšmė $(-1.671 > -2.89)$, tai  procesas turi vienetinę šaknį..}

\end{itemize}


\newpage
\section{Modelis}
\textnormal{Panaikinę komponenčių sezoniškumą, ir darkart atlikę vienetinės šaknies testus gauname, jog visos jos turi vienetinę šaknį. Pabandykime sudaryti tiesinį modelį be ankstinių ir patikrinti jo tinkamumą. R kvadratą gauname tik 0.5611. Patikrinkime, ar šio modelio liekanos sudaro BT}

\begin{figure}[H]
\centering 
\includegraphics[width=6cm,  height=6cm]{modelis1}
\caption{Liekanos nesudaro BT}
\end{figure}


\textnormal{Taigi, liekanos nėra stacionarios, štai ir mūsų ne itin gero modelio priežastis. Taigi, toliau bandysime sudaryti modelį liekanoms ir vėl tikrinti modelio tinkamumą }


\newpage
\section{Duomenų analizės išvados}
\textnormal{\             \ Remiantis duomenų analizės rezultatais, numatomas DGP(duomenis generuojančio proceso) tobulinimas.\\ Laikines sekas, turinčias vienetinę šaknį, būtina pakeisti į skirtumus, prireikus, infliaciją nagrinėti kaip laikinę seką,  t.y. sudarant regresiją nuo jos pačios, tačiau taip pat atsižvelgiant į vienetinės šaknies  įtaką regresijos rezultatams.\\ Mūsų manymu, šis infliacijos inercijos modelis leis ne tik abstrakčiai išnagrinėti šį reiškinį, bet ir taikyti epmirinę analizę, siekiant nustatyti infliacijos stabilizavimo veiksmingumą}

\end{document}
