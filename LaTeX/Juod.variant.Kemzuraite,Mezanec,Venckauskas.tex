\documentclass[a4paper]{article}

\usepackage[utf8]{inputenc}
\usepackage[L7x]{fontenc}
\usepackage[lithuanian]{babel}
\usepackage{amsmath}
\usepackage{graphicx}
\usepackage{graphics}
\usepackage{hyperref}
		
\begin{document}

\begin{titlepage}

\vskip 20pt
\centerline{\bf \large VILNIAUS UNIVERSITETAS}
\bigskip
\centerline{\large \textbf{MATEMATIKOS IR INFORMATIKOS FAKULTETAS}}
\bigskip
\centerline{\large \textbf{EKONOMETRINĖS ANALIZĖS KATEDRA}}
\vskip 120pt
\centerline{\bf \Large \textbf{Kursinis projektas}}
\vskip 50pt

\begin{center}
{\bf \LARGE Lietuvos infliacijos modelis}
\end{center}

\bigskip
\begin{center}
\large Alma Mežanec, \\
\large Daiva Kemzūraitė,	\\
\large Laurynas Venčkauskas\\
\bigskip
\large Ekonometrija, III kursas
\end{center}
\vskip 90pt
\begin{flushleft}
\large Kursinio projekto vadovas: \textbf{doc. dr. Remigijus Lapinskas}\\
\large Vadovo parašas \underline{\hskip 114pt}\\
\large Projekto įteikimo data \underline{\hskip 114pt}
\end{flushleft}
\vfill
\centerline{\large \textbf{VILNIUS, 2011}}
\end{titlepage}

\tableofcontents
\newpage

\section{Įžanga}

\textbf{Infliacija} (inflation) - yra pinigų nuvertėjimas, kuris pasireiškia prekių ir paslaugų kainų kilimu. \\
Infliacija– tai ne bet koks kainų  kilimas, tai nėra tam  tikrų prekių arba jų grupių kainų lygis. Be to, tai ne vienkartinis kainų pakilimas, o nuolatos besitęsiantis reiškinys, trunkantis gana ilgai. Kai kurių prekių kainos net ir infliacijos sąlygomis gali išlikti nepakitusios (arba net gali sumažėti).\\
Požiūris į infliaciją, kaip ekonominį reiškinį įvairiais laikotarpiais buvo nevienodas. Anksčiau dominavo tezė, kad infliacija yra išimtinai destruktyvi jėga. Šią tezę paneigė Dž. Keinsas, kuris teigė, kad infliacija yra milžiniškas pozytivus potencialas, nes jai esant, nuvertėja pinigai ir jų kaupimas darosi betikslis, skatinamas vartojimas ir kartu ekonomikos augimas. Nesant infliacijos, kaupiami pinigai, jie įšaldomi ir susidarius tam tikroms aplinkybėms, gali iššaukti ekonominę krizę. Dabartiniu metu infliacija įgavo visuotinį pobūdį, tapo įprastu reiškiniu ir viena iš opiausių socialinių ekonominių problemų Lietuvoje. Kainos dažniausiai keičiasi didėjimo linkme. Tačiau kainų pokyčio laipsnis labai nevienodas atskirais laikotarpiais. Netinkamas kainų lygio pokyčio interpretavimas tiesiogiai susijęs su netinkamai priimamais sprendimais. Todėl jo įtaka svarbi visai makroekonomikai.\\
Šį pokytį ir tirsime, pasitelkę ketvirtinius 2001-2011 metų duomenis: Lietuvos vartotojų kainų indeksą, naftos kainą bei procentinį nedarbo lygį. Sudarę laikines sekas, kurios turi bent po  vieną ilgą“ekskursiją” aukštyn ir žemyn,  tirdami priklausomybę tarp jų pačių, kointegracijos sąryšius, formuosime Lietuvos infliacijos modelį geriausiai paaiškinantį pačią infliaciją bei leidžiantį kuo tiksliau atlikti jos prognozę pasirinktam žingsnių skaičiui. Duomenys - aliuzija į VECM.\\
Taigi, pagrindinis šio ekonometrinio darbo tikslas yra suformuoti tokį Lietuvos infliacijos modelį, kurio komponentės kuo tiksliau paaiškintų pastarųjų dešimties metų  infliacijos dinamiką, struktūrinus lūžius krizės metu Lietuvoje bei, remiantis praeitimi,  leistų daryti išvadas apie jos elgesio tendencijas ateityje.

\newpage
\section{Duomenų analizė}

	\subsection{Būsimo modelio komponenčių teorinis ryšys su infliacija}
	
		\subsubsection{Infliacija ir vartotojų kainų indeksas}
		
Infliacijai išmatuoti Lietuvoje naudojamas vartotojų kainų indeksas (VKI). Vartotojų kainų indeksas (VKI) – parodo vartojimo prekių ir paslaugų, kurias įsigyja, už kurias sumoka ir kurias namų ūkiai panaudoja tiesiogiai patenkinti vartojimo poreikius, vidutinį kainų lygio pokytį per tam tikrą laikotarpį,Tai yra pagrindinis infliacijos rodiklis, rodantis vartojimo prekių ir paslaugų, kurias įsigyja, už kurias sumoka ir kurias namų ūkiai panaudoja tiesiogiai patenkinti vartojimo poreikius, vidutinį kainų lygio pokytį per tam tikrą laikotarpį. VKI Lietuvoje skaičiuojamas nuo 1992 m. gegužės mėn. Vartotojų kainų indeksas neapima prekių ir paslaugų, skirtų gamybai, pelno gavimui, kapitalo formavimui.\\
Pagrindinė informacinė bazė VKI skaičiavimui yra duomenys apie prekių ir paslaugų reprezentančių kainas, tarifus ir gyventojų išlaidas vartojimo prekėms ir paslaugoms įsigyti. Pagrindinis kainų informacijos šaltinis yra prekių ir paslaugų reprezentančių kainų ir tarifų registravimas atrinktose visų nuosavybės formų prekybos ir paslaugų sferos įmonėse. Kiekvienos prekės ar paslaugos kainų lygio pokytis daro skirtingą įtaką bendrajam VKI. Tai lemia išlaidų kiekvienai prekei ar paslaugai dalis bendroje gyventojų piniginių vartojimo išlaidų struktūroje. Skaičiuojant VKI vartojimo prekių ir paslaugų kainų santykiai atsveriami naudojant išlaidų prekėms, paslaugoms lyginamuosius svorius bendrose namų ūkių vartojimo išlaidose. Namų ūkių biudžetų tyrimo apie pinigines gyventojų vartojimo išlaidas duomenys yra pagrindinis informacijos šaltinis svoriams rengti.

\newpage \subsubsection{Infliacija ir nedarbas}
		
Ryšį tarp dviejų pagrindinių makroekonomikos rodiklių – nedarbo ir infliacijos -  parodo Filipso kreivė. Olbanas Filipsas 1958 m. paskelbtame darbe įrodė, kad tarp nedarbo lygio ir nominalaus darbo užmokesčio kilimo tempų yra atvirkštinė priklausomybė.\\
\begin{center} \includegraphics[scale=0.5]{Filipso} \end{center}
Nedarbo – infliacijos dilema (atvirkštinė priklausomybė) gali nustoti egzistavusi, jeigu infliacijos lūkesčiai yra racionalūs, t.y. pagrįsti ateities numatymu. Atkreipiame dėmesį į svarbiausius šios teorijos teiginius, nagrinėjamo klausimo pažiūriu. Racionalūs lūkesčiai – tai geriausia ateities prognozė, paremta visa turima informacija. Jei lūkesčiai racionalūs, galima teisingai prognozuoti infliacijos tempą. Ankstesnis infliacijos lygis neturi reikšmės, todėl nėra ir inercinės, užprogramuotos infliacijos. Infliacijos ateities prognozės pagrįstos ne ankstesnių laikotarpių duomenimis, bet būsimąja valstybės ekonomine politika, numatomais politikų veiksmais. Norint efektyviai prognozuoti, būtina žinoti galimus jų veiksmus. (Remiantis šia idėja, tikrinsime modelio tinkamumą. t.y ar nukirptų duomenų prognozė tiksliai atspindi tikrąsias infliacijos reikšmes)

\newpage \subsubsection{Infliacija ir naftos kainos}
Kuro kainų didėjimo poveikis - vienas pagrindinių infliaciją skatinančių veiksnių. Naftos kainų šokas veikia asimetriškai. Pasaulyje, senkant naftos ištekliams, tačiau tuo pat metu augant jos vartojimui, ypač dėl tolimųjų rytų augančios ekonomikos, neišvengiamai kyla jos kainos. Taigi, duomenyse matomas spartus naftos kainos kilimas susiformavodėl energetikos išteklių kainų didėjimo.  Naftos brangimas įtakoja greitesnį Lietuvos VKI didėjimą nė infliacijos dėl:

\begin{itemize}
\item didesnio kuro svorio Lietuvoje (besivystanti šalis, daugiau išlaidų “būtinybėms”);
\item mažesnio specifinio (ad quantum) akcizo degalams (nes harmonizacija dar nebaigta).
\end{itemize}


Taigi, teoriškai išnagrinėję kiekvienos komponentės įtaką infliacijai, galime nesunkiai pastebėti, kad  jos gan glaudžiai koreliuoja ir tarpusavyje. Egzogeniniu kintamuoju laikysime VKI logaritmų skirtumą, o endogeniais,- logaritmuotus nedarbo lygio bei naftos kainų duomenis. 2008 metų  pabaiga - 2010 metai - ekonominė krizė Lietuvoje. Duomenys tai puikiai atspindi: didėjantis kainų lygio pokytis, vis kylanti naftos kaina bei didėjantis nedarbas. Modelio parinkimas šiems duomenims - procesas, reikalaujanatis nuodugnios duomenų analizės.
\newpage
	\subsection{ECB modelis (remiantis Brouwner ir Ericsson modeliu (1998))}
		
Modeliuojant Lietuvos ekonomikos infliaciją, remsimės Europos centrinio banko (toliau - ECB) viešinamu, 2004 metų, vasario mėnesio, Nr. 306 straipsnio „A markup model of inflation of  the Euro area“ Europos Sąjungos šalių infliacijos modeliu. Jo  pagrindas -  Brouwner ir Ericsson (1998) modelis, aiškinantis Australijos šalies infliaciją, remiantis šalies vartotojų kainų indekso duomenimis. Žinoma, rastasis modelis buvo pagrindas tolimesniems veiksmams. Remiantis ECB infliacijos modelio sudarymo etapais, ir atsižvelgiant į tai, kokių Lietuvos makroekonominių kintamųjų duomenys yra laisvai prieinami ir kurie iš jų gali padėti paaiškinti infliacijos svyravimus, savo infliacijos modelį apsibrėžėme taip:

\begin{figure}[hc]
\centering
$\text{VKI}_t = \Psi \text{OIL}^\beta_t \text{NED}^\alpha_t$
\caption{[1],}
\end{figure}

\noindent kur

\begin{itemize}
\item \textit{VKI} - vartotojų kainų indeksas (2005 metais VKI = 100);
\item \textit{OIL} - naftos kaina šalies vidaus valiuta (litais);
\item \textit{NED} - nedarbas, išreikštas procentiniu dydžiu, t.y. nedirbančių žmonių procentas tarp ekonomiškai aktyvių gyventojų;
\item \textit{$\alpha$,$\beta$,$t$} - konstantos, iš realiųjų skaičių aibės.
\end{itemize}
 


\noindent Log-tiesinė ([1]) lygties išraiška:

\begin{figure}[!h]
\centering
$\text{vki}_t = \log{\Psi} + \beta \text{oil}_t + \alpha \text{ned}_t$
\caption{[2],}
\end{figure}

\noindent kur  kintamieji, užrašyti mažųjų raidžių kombinacijomis, žymi atitinkamų kintamųjų, žymimų didžiųjų raidžių kombinacijomis, logaritmus.

\newpage	
	\subsection{Lietuvos duomenys modelyje}
	
Atsižvelgiant į užsibrėžtą tikslą, Lietuvos ekonomiką paaiškinančios lygties ([1]) pagalba, šiame kursiniame darbe naudojami šie Lietuvos makroekonominiai kintamieji: vartotojų kainų indeksas, nedarbas ir naftos kaina. Vartotojų kainų indekso mėnesinius ir nedarbo ketvirtinius duomenis parsisiuntėme iš Lietuvos statistikos departamento, o naftos kainos duomenis šalies vidaus valiuta gauti buvo šiek tiek sunkiau: iš pražių parsisiuntėme naftos kainos duomenis doleriais už barelį iš Federalinio rezervų banko tinklalapio, o po to šią kainą konvertavome į litus už barelį, pasinaudodami Lietuvos banko svetainėje skelbtais valiutų kursais (LTL/USD). Kadangi Lietuvos infliacijos modeliavimui naudojami ketvirčių duomenys nuo 2001 metų ketvirtojo ketvirčio iki 2011 metų antrojo ketvirčio, turimus vki duomenis konvertavome į ketvirtinius, suskaičiuodami trijų mėnesių duomenų vidurkius, t.y. trijų mėnesių vidurkis (pradedant nuo 2001 10-12) atitinka vieną ketvirtį. Paveikslėlyja galime pamatyti pirminių duomenų: vartotojų kainų indekso (VKI), naftos kainos (OIL), nedarbo (NED), ir infliacijos (inf=diff(log(vki)), apibrėžtos kaip vartotojų kainų indekso logaritmuotų duomenų skirtumas, t.y. vartojimo prekių ir paslaugų kainų indekso procentinis pokytis per ketvirtį.

\begin{figure}[hc]
\centering
\includegraphics[scale=0.5]{All_Data}
\caption{}
\end{figure}

\newpage \subsubsection{Vartotojų kainų indeksas}

Tirdami infliaciją, naudojamės statistikos departamento skelbtais vki duomenimis. Ekonometrinėje analizėje remiamės transformuotais vki duomenimis, t.y. toliau laikome, jog  $\text{vki} = \log(\text{vki})$. Duomenų transformacija atlikta, remiantis [1] lygties reikalavimais: multiplkatyviąją lygtį keičiant adityviąja logaritmuojant, t.y. iš ([1]) gauname ([2]) lygtį, pirmąją išlogaritmavus.\\ Žemiau esančiame grafike galime pamatyti vki duomenų grafiką:

\begin{figure}[!h]
\centering
\includegraphics[scale=0.5]{Lvki}
\caption{logaritmuotas vartotojų kainų indeksas}
\end{figure}
 
\noindent Iš grafiko įžvelgiamas duomenų tendencingumas, t.y. teigiamas trendas, kitaip, – augimas. Taip pat matome, jog šį procesą būtų sunku pavadinti atsitiktiniu stacionariu procesu su trendu (trend stationary), kadangi įžvelgiamas didelis priklausomumas nuo ankstinių, .t.y. procesą įtakoja tai, kokios reikšmės buvo praeityje. Labai pravartu patikrinti, ar iš tikrųjų taip nėra, kadangi,  jei vki procesas turi vienetinę šaknį, tuomet jis negali be papildomų sąlygų arba pakeitimų būti naudojamu modelio sudarymui. Pastarąjį teiginį tikriname remiantis Dickey-Fuller testu, kuris tikrina, ar teisinga mūsų hipotezė, jog vki turi vienetinę šaknį, ar ji yra klaidinga. Rezultatai patvirtina mūsų hipotezę, jog procesas vki turi vienetinę šaknį, kadangi testo gauta statistika yra didesnė už 5\% kritinę reikšmę :

\begin{table}[!h]
\begin{center}
\begin{tabular}{rrr}
  \hline
 & Testo statistika & 5\% krit. r. \\ 
  \hline
1 & 1.92 & -1.95 \\ 
   \hline
\end{tabular}
\end{center}
\end{table}

\noindent Išvada:  naudoti vki, vertinant ([2]) lygtį, nebūtų korektiška. Tačiau tolimesni veiksmai labai priklausytų nuo to, ar kiti kintamieji yra stacionarūs, ar taip pat turi vienetinę šaknį.

\newpage \subsubsection{Naftos kainos} 
		
Lygtyje ([2]) vienas iš paaiškinamųjų kintamųjų yra logaritmuoti naftos kainos šalies valiutos duomenys, tekste žymimi trumpiniu oil. Duomenų transformacija (logaritmavimas) atlikta, remiantis naudojamos lygties reikalavimais, multiplikatyviąją lygtį keičiant adityviąja logaritmuojant, t.y. iš ([1]) gauname ([2]) lygtį, pirmąją išlogaritmavus.\\ Žemiau esančiame grafike galime pamatyti oil duomenų grafiką:

\begin{figure}[!h]
\centering
\includegraphics[scale=0.5]{Loil}
\caption{logaritmuota naftos kaina litais}
\end{figure}

\noindent Grafike, tyrinėjamu laiko periodu, įžvelgiama naftos kainos augimo tendencija. Tačiau, ar oil procesas yra stacionarus procesas apie tiesę, patikrinsime jau mūsų minėtu Dickey-Fuller testu, pagal kurį priimama arba atmetama hipotezė, jog procesas turi vienetinę šaknį, atsižvelgiant į gautą testo statistikos ir kritinę reikšmes. Rezultatai patvirtina hipotezę: procesas oil turi vienetinę šaknį, kadangi testo gauta statistika yra didesnė už 5\% kritinė reikšmę :

\begin{table}[!h]
\begin{center}
\begin{tabular}{rrr}
  \hline
 & Testo statistika & 5\% krit. r. \\ 
  \hline
1 & -1.77 & -2.93 \\ 
   \hline
\end{tabular}
\end{center}
\end{table}

\noindent Taigi, turime, jog vki ir oil procesai turi vienetinę šaknį.
		
\newpage \subsubsection{Nedarbo lygis}

Lygtyje ([2]) antrasis iš paaiškinamųjų kintamųjų yra nedarbo lygio logaritmuoti duomenys, tekste žymimi trumpiniu ned. Duomenų transformacija (logaritmavimas), kaip ir kitų kintamųjų atveju, atlikta remiantis naudojamos lygties reikalavimais,  multiplikatyviąją lygtį keičiant adityviąja logaritmuojant, t.y. iš ([1]) gauname ([2]) lygtį, pirmąją išlogaritmavus.\\ Žemiau esančiame grafike galime pamatyti ned sudaromo proceso grafiką:

\begin{figure}[!h]
\centering
\includegraphics[scale=0.5]{Lned}
\caption{logaritmuotas nedarbo lygis procentais}
\end{figure}

\noindent Iš šio kintamojo grafiko galime padaryti išvadą, jog procesas nėra tendencingas, tačiau kiekvieno sekančio periodo reikšmė labai priklauso nuo prieš tai buvusio laiko momento reikšmės, kadangi matome ilgas ekskursijas žemyn ir aukštyn. Taigi,  vėl turime patikrinti hipotezę dėl vienetinės šaknies egzistavimo, vadovaudamiesi ne kartą tekste minėtu Dickey-Fuller testu. Rezultatai mums ir vėl leidžia priimti hipotezę dėl vienetinės šaknies egzistavimo, kadangi testo statistika yra didesnė už 5\% kritinę reikšmę:

\begin{table}[!h]
\begin{center}
\begin{tabular}{rrr}
  \hline
 & Testo statistika & 5\% krit. r. \\ 
  \hline
1 & -0.27 & -1.95 \\ 
   \hline
\end{tabular}
\end{center}
\end{table}

\noindent Gauti rezultatai rodo, jog turime tris procesus su vienetinėmis šaknimis, su kurių pagalba siekiame aiškinti Lietuvos infliacijos procesą.

\newpage \subsubsection{Infliacija} 

Infliacija – šio kursinio darbo tikslas, taigi būtinai turime atlikti ir šių duomenų analizę. Jau žinome, jog vki - procesas, kuriuo remiantis apskaičiuosime istorinius infliacijos duomenis. Tai nėra stacionarus procesas, o pati infliacija, apibrėžiama  kaip vki proceso skirtumai, t.y. procentinis kainų lygio pokytis :

\begin{table}[!h]
\centering
$\text{inf}_t = \log(VKI_t) - \log(VKI_{t-1}) = vki_t - vki_{t-1}$
\caption{}
\end{table}

\noindent Žemiau matome inf proceso grafinį vaizdą:

\begin{figure}[!h]
\centering
\includegraphics[scale=0.5]{inf}
\caption{infliacija}
\end{figure}

\noindent Grafikas, atvaizduojantis inf procesą, atrodo pakankamai stochastinis, nors, kai kuriais atvejais, pastebimas inertiškumas, t.y. priklausymas nuo praeito periodo reikšmių. Kadangi vki procesas turėjo vienetinę šaknį, ir šiame vki skirtumų procese įžvelgiamas šioks toks inertiškumas. Atliksime proceso inf vienetinės šaknies testą, iškeldami hipotezę, jog procesas inf turi vienetinę šaknį. Dickey-Fuller testo rezultatai mums neleidžia atmesti vienetinės šaknies hipotezės:

\begin{table}[!h]
\begin{center}
\begin{tabular}{rrr}
  \hline
 & Testo statistika & 5\% krit. r. \\ 
  \hline
1 & -1.39 & -1.95 \\ 
   \hline
\end{tabular}
\end{center}
\end{table}

\noindent Taigi, vki proceso skirtumai, atitinkantys mūsų tiriamąją infliaciją, vis dar turi vienetinę šaknį. 

\newpage \subsubsection{Infliacijos ir modelio komponenčių sąryšių tyrimas}

\noindent Šiuo atveju, galėtume drąsiai paieškoti kointegracijos sąryšių tarp inf, oil ir ned kintamųjų, kadangi kiekvienas iš jų turi vientinę šaknį. Išnagrinėkime grafiškai, kaip tarpusavyje sąveikauja šių trijų kintamųjų grafikai (vki grafiko nenagrinėjame, kadangi komponentės visas naudingumas jau išnaudotas – turime susikonstravę Lietuvos infliacijos duomenų laikinę seką): 

\begin{figure}[!h]
\centering
\includegraphics[scale=0.5]{Ldata}
\caption{bendras duomenų grafikas}
\end{figure}

\noindent Pradėkime nagrinėti kiekvieną iš kintamųjų porų sąryšių atskirai.

\newpage
\subsubsection*{Infliacija ir naftos kaina}

\noindent Žemiau esančiame lange galime pamatyti grafinį inf ir oil kintamųjų sąveikos vaizdą:

\begin{figure}[!h]
\centering
\includegraphics[scale=0.5]{inf_oil}
\caption{infliacija ir naftos kaina}
\end{figure}

Grafike gana aiškiai išryškėja kintamųjų oil ir inf darnus judėjimas. Abu kintamieji išlaiko tą patį tendencingumą;  kiekvieno kintamojo pakilimo ir kritimo periodai sutampa.\\ Pažvelkime į šių kintamųjų sudarytos tiesinės regresijos rezultatus:

\begin{table}[!h]
\begin{center}
\begin{tabular}{cccc} 
$\text{inf}_t =$ & $-1,256*10^{-2}$ & $+$ & $1,264*10^{-4}*\text{oil}_t$ \\ 
  & (0.017735) && (0.000113) \\ 
\end{tabular} 
\end{center}
\end{table}


\noindent Regresijos rezultatai dar kartą patvirtina, jog naftos kaina oil pakankamai tiksliai paaiškina infliaciją, kadangi koeficientas prie oil yra labai reikšmingas (skliausteliuose, po įvertintu koeficientu, matome jo p-reikšmę), be to, gautos modelio liekanos yra stacionarios, t.y, jog šias sekas sieja kointegracijos ryšys, o iš to seka ilgalaikės pusiausvyros sąryšis. Taigi, turint kointegruotas sekas inf ir oil, galime teigti, jog mūsų vertinamoji lygtis ([2]) perauga į paklaidų korekcijos lygtį.

\newpage
\subsubsection*{Infliacija ir nedarbas}

Žemiau esančiame grafike, galime pamatyti inf ir ned kintamųjų sąveikos vaizdą:

\begin{figure}[!h]
\centering
\includegraphics[scale=0.5]{inf_ned}
\caption{infliacija ir nedarbo lygis}
\end{figure}

\noindent Iškyla aliuzija į neigiamą priklausomybę, t.y. infliacijai augant, nedarbas mažėja. Prisiminus pradžioje aptartą Fillips'o kreivės (Phillips curve) teoriją, mūsų esamasis ryšys nestebina,- dar kartą pasitvirtina XIXa. atrasta teorija dėl nedarbo ir infliacijos priklausomybės. Belieka patikrinti svarbų faktą,- sekų inf ir ned kointegruorumą, sudarant regresiją tarp laikinių sekų:

\begin{center}
\begin{tabular}{cccc} 
$\text{inf}_t =$ & $0,019726$ & $-$ & $0,001069*\text{ned}_t$ \\ 
  & $(4,9*10^{-5})$ && (0.00637) \\ 
\end{tabular} 
\end{center}

\noindent Regresijos rezultatai patvirtina Fillips'o kreivės (Phillips curve) teoriją bei mūsų spėjimą, jog nedarbas pakankamai neblogai paaiškina infliacijos vyksmus,- koeficientas prie nedarbo kintamojo ned yra reikšmingas(skliausteliuose, po įvertintu koeficientu, matome jo p-reikšmę) ir jo reikšmė yra neigiama, be to, tiesinio modelio liekanos tarp kintamųjų inf ir ned yra stacionarios, o tai parodo, jog inf ir ned laikinės sekos yra kointegruotos, t.y. susietos ilgalaikės pusiausvyros sąryšiu.\\
Taigi, šio skyrelio svarbiausia išvada yra ta, jog mūsų kintamuosius, kurie turi vienetines šaknis ir, be to, tarp kurių iš viso yra du kointegruotumo sąryšiai, reikėtų aprašyti paklaidų korekcijos modeliu, kuriame įtraukiame paklaidų korekcijos narį, t.y. kointegracijos sąryšių liekanas. Turime daugiau nei du kointegracijos sąryšius bei norime sudaryti modelį tinkamą infliacijos prognozei,- tolesnią analizę atliksime, sudarydami vektorinį autoregresinį modelį su paklaidų korekcijos nariu (VECM), kurį vėliau parametrizuosime į vektorinį autoregresinį (VAR) modelį, patogiausią infliacijos prognozei.


\newpage
\section{Lietuvos infliacijos modelis. Modelio ekonometrinė analizė}
	\subsection{Modelio formos pasirinkimas}

Infliacijos modeliavimui pasirinkome vektorinio autoregresinio (VAR) modelio formą. Tikslas: VAR naudojimas prognozei. Pastarąją lyginsime su turimais duomenimis ir analizuosime, ar mūsų modelis gali tinkamai įvertinti ateities infliaciją.

	\subsection{Modelio sudarymas}

Lietuvos duomenys turi vienetinę šaknį, todėl sudarysime modelį skirtumams.  Pirmiausia patikrinsime, ar turimi kintamieji nėra kointegruoti. Jei rasime kintamųjų kointegruotumą, - modelį papildysime paklaidų korekcijos nariu.\\
Modelio ankstinių eilę parenkame, vadovaudamiesi Švarco informaciniu kriterijumi (Schwarz information criterion). Išvada:  procesą geriausiai aprašo VAR su dviem ankstiniais, t.y. dabartinė kintamojo reikšmė priklauso nuo dviejų ankstesnių reikšmių.\\
Kitas žingsnis, sudarant modelį, yra patikrinti kointegravimo sąryšių skaičių. Tai atliekame, naudodami Johanseno tikrinių matricos reikšmių testą. Iš čia seka du kointegraviomo sąryšiai, -  į modelį turėsime įtraukti paklaidų korekcijos narį.\\Sekantis žingsnis - įvertinti VAR modelio (su paklaidų korekcijos nariu) koeficientus. Turėdami juos, transformuojame VEC modelį į VAR lygiams.
\newpage
	\subsection{Įvertintas modelis}
Atlikę praeitame skyrelyje aprašytas procedūras, gauname tokį modelį:

\vspace{3mm}
$\begin{cases}
\text{inf}_{t} = C_{1} + \alpha_{1} T + \beta_{1,1} \text{inf}_{t-1} + \beta_{1,2} \text{inf}_{t-2} + \beta_{1,3} \text{oil}_{t-1} + \beta_{1,4} \text{oil}_{t-2} +  \beta_{1,5} \text{ned}_{t-1} + \beta_{1,6} \text{ned}_{t-2}\\
\text{oil}_{t} = C_{2} + \alpha_{2} T + \beta_{2,1} \text{inf}_{t-1} + \beta_{2,2} \text{inf}_{t-2} + \beta_{2,3} \text{oil}_{t-1} + \beta_{2,4} \text{oil}_{t-2} +  \beta_{2,5} \text{ned}_{t-1} + \beta_{2,6} \text{ned}_{t-2}\\
\text{ned}_{t} = C_{3} + \alpha_{3} T + \beta_{3,1} \text{inf}_{t-1} + \beta_{3,2} \text{inf}_{t-2} + \beta_{3,3} \text{oil}_{t-1} + \beta_{3,4} \text{oil}_{t-2} +  \beta_{3,5} \text{ned}_{t-1} + \beta_{3,6} \text{ned}_{t-2}\\
\end{cases}$
\vspace{3mm}

Konstantas $C_i$, trendo koeficientus $\alpha_i$ ir kintamųjų \textit{vki}, \textit{oil},  \textit{ned} koeficientus $\beta_{i,j}$ galite rasti žemiau pavaizduotose lentelėse.

\begin{table}[ht]
 \begin{center}
 \begin{tabular}{llll}\hline\hline
\multicolumn{1}{l}{$M_1$}&\multicolumn{1}{c}{$\text{inf}_{t-1}$}&\multicolumn{1}{c}{$\text{oil}_{t-1}$}&\multicolumn{1}{c}{$\text{ned}_{t-1}$}\tabularnewline
\hline
$\text{inf}_{t}$&$ 4,122 * 10^{-1}$&$ 3,281 * 10^{-5}$&$-1,758 * 10^{-3}$\tabularnewline
$\text{oil}_{t}$&$        1554,585$&$ 7,593 * 10^{-1}$&$-8,982$\tabularnewline
$\text{ned}_{t}$&$        -72,245$&$-7,132 * 10^{-3}$&$ 1,372$\tabularnewline
\hline
 \end{tabular}
 \end{center}
\end{table}

\begin{table}[ht]
 \begin{center}
 \begin{tabular}{llll}\hline\hline
\multicolumn{1}{l}{$M_2$}&\multicolumn{1}{c}{$\text{inf}_{t-2}$}&\multicolumn{1}{c}{$\text{oil}_{t-2}$}&\multicolumn{1}{c}{$\text{ned}_{t-2}$}\tabularnewline
\hline
$\text{inf}_{t}$&$-3,367 * 10^{-1}$&$ 5,371 * 10^{-5}$&$ 5,733 * 10^{-4}$\tabularnewline
$\text{oil}_{t}$&$        -740,575$&$-5,423 * 10^{-1}$&$ 7.594$\tabularnewline
$\text{ned}_{t}$&$          -7,623$&$ 1,469 * 10^{-2}$&$-4,738 * 10^{-1}$\tabularnewline
\hline
 \end{tabular}
 \end{center}
\end{table}

\begin{table}[ht]
 \begin{center}
 \begin{tabular}{lll}\hline\hline
\multicolumn{1}{l}{$M$}&\multicolumn{1}{c}{C}&\multicolumn{1}{c}{T}\tabularnewline
\hline
$\text{inf}_{t}$&$ 1,961 * 10^{-3}$&$2,368 * 10^{-4}$\tabularnewline
$\text{oil}_{t}$&$74,276$&$3,186$\tabularnewline
$\text{ned}_{t}$&$ 1,544 * 10^{-1}$&$2,004 * 10^{-2}$\tabularnewline
\hline
 \end{tabular}
 \end{center}
\end{table}

\subsection{Rezultatų analizė}

Kaip ir buvo galima tikėtis, koeficientai infliacijos lygtyje, prie naftos kainos, yra teigiami, t.y. naftos kainų kilimas dvejuose praeituose ketvirčiuose didina infliaciją. Nedarbo įtaka infliacijai yra atvirkštinė t.y. didėjant nedarbui, mažėja infliacija. Galų gale, pačios infliacijos priklausomybė nuo ankstesnių reikšmių yra tiesioginė t.y. infliacijos kilimas ankstesniais periodais, didina infliaciją šiame periode.
\newpage
\section{Lietuvos infliacijos modelio prognozė}

Turint modelį, tenkinantį prognozavimo sąlygas, galime numatyti infliacijos elgsenos ateitį.\\
Norėdami ištirti sudaryto modelio naudingumą, pirmiausia jį sudarėme, nepanaudodami paskutinių keturių ketvirčių duomenų. Taigi, pasinaudodami prieš tai minėtomis prielaidomis, atliekame keturių žingsnių priekin prognozę.\\ Stebėtos ir prognozuotos infliacijų elgsenos tendencijos:

\begin{figure}[!h]
\centering
\includegraphics[scale = 0.5]{inf_fcast}
\caption{infliacijos prognozė}
\end{figure}

\subsection{Modelio prognozės apibendrinimas}
\noindent Taigi, infliacija prognozuojama mažesnė, tačiau turinti didėjimo tendenciją, - ryškus pagrindinis struktūrinis pasikeitimas išlieka.
\newpage
\section{Išvados}

Infliacija - yra sudėtingo politikos ir ekonomikos sąveikos visuomenėje proceso išdava. Tokiame kontekste priežasties - pasekmės kalba lengvai tampa klaidinančia. Už kainų lygio kilimo slepiasi nesuskaičiuojamų firmų ir atskirų asmenų priimami įkainojimų sprendimai. Tam tikra prasme šie sprendimai yra tiesioginė infliacijos priežastis. Tačiau kainų keitimo sprendimai yra paremti rinkos signalais ir politine situacija šalyje. O šie, savo ruožtu, yra kitų priežasčių grandinės pasekmė.\\
Šiame darbe, tirdami sąsajas tarp duomenų, jų reikšmingumą, įvertinant infliacijos priklausomybę nuo jų, remdamiesi praeities įtaka, sukonstravome Lietuvos infliacijos modelį (VECM), apie kurio gerumą byloja ir liekanos, sudarančios BT. Pastarasis modelis iš visų nagrinėtųjų geriausiai atspindi pagrindinius ekonomikoje vykstančius procesus (infliacijos struktūrinius lūžius).  Todėl būtina pastebėti, jog naftos kainos ir procentinis nedarbas pasirinkti tikslingai t.y jų reguliavimo metodai turės tiesioginę įtaką infliacijai. Taigi, infliacijos reguliavimo metodai bus efektyvūs tik tada, kai jie adekvačiai atitiks jos priežastis. Vienas iš pagrindinių išsikeltų uždavinių - infliacijos prognozė- (VECM $\longrightarrow$ VAR), išspręstas gana tikslingai,- racionalių lūkesčių pasekmė.\\ 
Taigi, mūsų sukurtas Lietuvos infliacijos modelis -  priemonė, priimant makroekonominius sprendimus, leidžianti išvengti grubių infliacijos „spėjimo“ klaidų bei netinkamų investicijų pasekmių. 

\newpage
\section{Literatūros sąrašas}

\begin{itemize}
\item O. Blanchard  „Makroekonomika“;
\item \url{www.stat.gov.lt}; 
\item \url{www.lb.lt};
\item \url{http://research.stlouisfed.org/fred2/series/OILPRICE/downloaddata};
\item \url{www.ecb.int/pub/pdf/scpwps/ecbwp306.pdf}.
\end{itemize}

\newpage
\section{Priedai}
\begin{itemize}
\item{Literatūros sąraše minimos internetinės svetainės - Lietuvos infliacijos modelyje naudotų duomenų šaltiniai.}\\
\item{Paskutinioji nuoroda - straipsnis apie Brouwner ir Ericsson modelį Ausralijos infliacijai (1998).}
\end{itemize}
\end{document}
