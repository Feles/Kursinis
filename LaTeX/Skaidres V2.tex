\documentclass[utf8x,hyperref={unicode}]{beamer}
\mode<presentation>
\usetheme{warsaw}
\setbeamertemplate{navigation symbols}{} 
\setbeamertemplate{footline}
{%
    \leavevmode%
    \hbox{\begin{beamercolorbox}[wd=.5\paperwidth,ht=2.5ex,dp=1.125ex,leftskip=.3cm 	plus1fill,rightskip=.3cm]{author in head/foot}%
    \usebeamerfont{author in head/foot}\insertshortauthor
    \end{beamercolorbox}%
    \begin{beamercolorbox}[wd=.5\paperwidth,ht=2.5ex,dp=1.125ex,leftskip=.3cm,rightskip=.3cm plus1fil]{title in head/foot}%
    \usebeamerfont{title in head/foot}\insertshorttitle \hfill p.
	\insertpagenumber\enspace iš \insertdocumentendpage\enspace
    \end{beamercolorbox}}%
  \vskip0pt%
}

\usepackage[L7x]{fontenc}
\usepackage[lithuanian]{babel}
\usepackage{lmodern}
\usepackage{amsmath}
\usepackage{amssymb}
\usepackage{bm}
\usepackage{graphicx}


    \title{Lietuvos infliacijos modelis}
   \author{Daiva Kemzūraitė
      \and Alma Mežanec
      \and Laurynas Venčkauskas}
\institute{Vadovas:    doc. dr. R. Lapinskas
      \and Vilniaus Universitetas
      \and Matematikos ir informatikos fakultetas
\and Ekonometrinės analizės katedra
      \and Ekonometrija, 3 kursas}

\begin{document}
%------------------------------------------------------------------------------------------------------------------------
%                    1
%------------------------------------------------------------------------------------------------------------------------
\begin{frame}
	\titlepage
\end{frame}
%------------------------------------------------------------------------------------------------------------------------
%                    2
%------------------------------------------------------------------------------------------------------------------------
\begin{frame}
\frametitle{Darbo tikslas}

\begin{center}
\item\Large\textbf{ Lietuvos infliacijos modelio sudarymas}
\end{center}

\end{frame}
%------------------------------------------------------------------------------------------------------------------------
%                    3
%------------------------------------------------------------------------------------------------------------------------
\begin{frame}
\frametitle{Pasirinktas teorinis modelis}

\textbf{Modifikuotas Brouwner ir Ericsson  modelis (1998)}

\begin{equation}
\text{VKI}_t = \Psi \text{OIL}^\alpha_t \text{NED}^\beta_t \text{LIBOR}^\gamma_t,
\end{equation}

\begin{itemize}
\item VKI - vartotojų kainų indeksas (2005 metais VKI = 100);
\item OIL - naftos kaina šalies vidaus valiuta (litais);
\item NED - nedarbas, išreikštas procentiniu dydžiu, t.y. nedirbančių žmonių procentas tarp ekonomiškai aktyvių gyventojų;
\item LIBOR - Londono tarpbankinė palūkanų norma;
\item $t$ - laikas;
\item $\Psi$,$\beta$,$\alpha$,$\gamma$ - realios konstantos.

\end{itemize}

\end{frame}
%------------------------------------------------------------------------------------------------------------------------
%                    4
%------------------------------------------------------------------------------------------------------------------------
\begin{frame}
\frametitle{}
\textbf{Log-log lygties išraiška:}


\begin{equation}
\textit{vki}_t = \psi + \alpha \textit{oil}_t + \beta \textit{ned}_t + \gamma \textit{libor}_t+\epsilon_t
\end{equation}

\textbf{Čia:} kintamieji, užrašyti mažųjų raidžių kombinacijomis, žymi atitinkamų kintamųjų, žymimų didžiųjų raidžių kombinacijomis, logaritmus.

\end{frame}
%------------------------------------------------------------------------------------------------------------------------
%                    5
%------------------------------------------------------------------------------------------------------------------------
\begin{frame}
\frametitle{Komponenčių logaritmų grafikai}

Vartotojų kainų indekso (\textit{vki}), naftos kainos (\textit{oil}), nedarbo (\textit{ned}), Londono tarpbankinės palūkanų normos (\textit{libor}) grafikai.

\begin{figure}[hc]
\centering
\includegraphics[scale=0.35]{All_Data}
\end{figure}
\end{frame}
%------------------------------------------------------------------------------------------------------------------------
%                    6
%------------------------------------------------------------------------------------------------------------------------
\begin{frame}
\frametitle{Duomenų analizė}

\textbf{Dickey-Fuller testų lentelė:}

\begin{table}[!h]
\begin{center}
\begin{tabular}{rrr}
  \hline
Kintamasis & Testo statistika & 5\% krit. r. \\
  \hline
\textit{vki} & -2.98 & -3.50 \\
\textit{oil} & -2.98 & -3.50 \\
\textit{ned} & -0.20 & -1.95 \\
\textit{libor} & -1.04 & -3.50 \\
   \hline
\end{tabular}
\end{center}
\end{table}

\end{frame}

%------------------------------------------------------------------------------------------------------------------------
%                    7
%------------------------------------------------------------------------------------------------------------------------
\begin{frame}

\frametitle{Išvados iš duomenų analizės}
\begin{itemize}
\item\large{Visi nagrinėti procesai turi vienetines šaknis.}
\item\large{Sudaromas Lietuvos infliacijos modelis.}
\end{itemize}
\end{frame}

%------------------------------------------------------------------------------------------------------------------------
%                    8
%------------------------------------------------------------------------------------------------------------------------
\begin{frame}
\frametitle{Modelis lygiams}
\begin{center}
$\textit{vki}_t = \psi + \alpha\textit{oil}_t + \beta\textit{ned}_t + \gamma\textit{libor}_t + \epsilon_t$ (2)
\end{center}

\textbf{Lygties (2) įvertinimas:}

\begin{center}
\begin{tabular}{cccccccccc} 
$\textit{vki}_t =$&$4,5$&$-0.11\times\text{oil}_t$&$-0.15\times\text{ned}_t$&$-0.12\times\text{libor}_t$&$+\hat{\epsilon}_t$\\ 
  &$(\sim 0)$&$(0,03)$&$(0,009)$&$1,22 \times 10^{-5}$\\ 
\end{tabular} 

\textit{Lygtis:} ($\hat{2}$)
\end{center}



\begin{itemize}
\item $\hat{\epsilon}_t$ - stacionarus procesas.
\item Johanseno tikrinių matricos reikšmių testas randa 3 kointegracijos sąryšius.
\end{itemize}

\end{frame}

%------------------------------------------------------------------------------------------------------------------------
%                    9
%------------------------------------------------------------------------------------------------------------------------
\begin{frame}
\frametitle{Modelis}\indent

\indent Modelyje (2) komponentės kointegruotos - nagrinėjamas paklaidų korekcijos modelis:

\begin{equation}
\Delta\textit{vki}_t = \psi + \alpha\Delta\textit{oil}_t + \beta\Delta\textit{ned}_t + \gamma\Delta\textit{libor}_t + \lambda\textit{e}_{t-1} + \epsilon_t
\end{equation}

\end{frame}

%------------------------------------------------------------------------------------------------------------------------
%                    10
%------------------------------------------------------------------------------------------------------------------------
\begin{frame}
\frametitle{$\Delta\textit{vki}$ stacionarumas}

\begin{itemize}
\item Dickey-Fuller testas gali rasti nestacionarumą stacionariuose duomenyse su struktūriniu lūžiu.
\item Apskaičiuojame konkrečią statistikos reikšmę $\Delta\textit{vki}$ duomenims.
\end{itemize}

\begin{table}[!h]
\begin{center}
\begin{tabular}{rrr}
  \hline
 & Testo statistika & 5\% krit. r. \\ 
  \hline
 & -4.8 & -4.3 \\ 
   \hline
\end{tabular}
\end{center}
\end{table}

\begin{itemize}
\item $\Delta\textit{vki}$ - stacionarus.
\end{itemize}

\end{frame}

%------------------------------------------------------------------------------------------------------------------------
%                    11
%------------------------------------------------------------------------------------------------------------------------
\begin{frame}
\frametitle{$\Delta\textit{vki}$ stacionarumas}



\includegraphics[scale=0.5]{Dvki}

\end{frame}

%------------------------------------------------------------------------------------------------------------------------
%                    11
%------------------------------------------------------------------------------------------------------------------------
\begin{frame}
\frametitle{$\Delta\textit{ned}$, $\Delta\textit{oil}$ ir $\Delta\textit{libor}$ stacionarumas}

\begin{figure}
\begin{minipage}[b]{0.32\linewidth}
\includegraphics[scale=0.175]{Dned}
\end{minipage}
\begin{minipage}[b]{0.32\linewidth}
\includegraphics[scale=0.175]{Doil}
\end{minipage}
\begin{minipage}[b]{0.32\linewidth}
\includegraphics[scale=0.175]{Dlibor}
\end{minipage}
\end{figure}

\begin{table}[!h]
\begin{center}
\begin{tabular}{llll}
  \hline
Kintamasis & Testo statistika & 5\% krit. r. & Stacionarus proc. \\ 
  \hline
$\Delta\textit{ned}$ & -2.8 & -1.95 & + \\
$\Delta\textit{oil}$ & -4.8 & -1.95 & + \\
$\Delta\textit{libor}$ & -2.7 & -1.95 & + \\ 
   \hline
\end{tabular}
\end{center}
\end{table}
\end{frame}

%------------------------------------------------------------------------------------------------------------------------
%                    12
%------------------------------------------------------------------------------------------------------------------------
\begin{frame}
\frametitle{Paklaidų korekcijos modelis}
\begin{itemize}
\item{Paklaidos korekcijos narys $e$ yra lygties ($\hat{2}$) paklaidos $\hat{\epsilon}_t$.}
\item Įvertiname (4) lygtį:
\end{itemize}
\begin{equation}
\begin{minipage}[b]{0.8\linewidth}
$\Delta\textit{vki}_t = \psi + \delta\textit{e}_{t-1} + \alpha_0\Delta\textit{oil}_t + \alpha_1\Delta\textit{oil}_{t-1} + \alpha_2\Delta\textit{oil}_{t-2} + \alpha_3\Delta\textit{oil}_{t-3} + \alpha_4\Delta\textit{oil}_{t-4} + \beta_0\Delta\textit{ned}_{t} + \beta_4\Delta\textit{ned}_{t-4} + \gamma_2\Delta\textit{libor}_{t-2} + \epsilon_t$
\end{minipage}
\end{equation}

\begin{table}[!h]
\begin{center}
\begin{tabular}{l|ccc} 
\hline
Kintamasis & Koeficiento įvertis & p-reikšmė\\
\hline
$\textit{e}_{t-1}$&0.114&0.03\\
$\Delta\textit{oil}_{t}$&0.04&0.007\\
$\Delta\textit{oil}_{t-1}$&0.037&0.02\\
$\Delta\textit{oil}_{t-2}$&0.041&0.02\\
$\Delta\textit{oil}_{t-3}$&0.044&0.005\\
$\Delta\textit{oil}_{t-4}$&0.04&0.008\\
$\Delta\textit{ned}_{t}$&0.017&0.002\\
$\Delta\textit{ned}_{t-4}$&-0.069&0.0001\\
$\Delta\textit{libor}_{t-2}$&-0.013&0.15\\
\hline
\end{tabular} 
\end{center}
\end{table}
\end{frame}

%------------------------------------------------------------------------------------------------------------------------
%                    13a
%------------------------------------------------------------------------------------------------------------------------
\begin{frame}
\frametitle{Įvertinta ir faktinė infliacija}

\begin{figure}[h!]
\center
\includegraphics[scale=0.49]{inf_fit2}
\end{figure}

\end{frame}
%------------------------------------------------------------------------------------------------------------------------
%                    13b
%------------------------------------------------------------------------------------------------------------------------
\begin{frame}
\frametitle{Modelio tinkamumo tikrinimas}

\begin{figure}[h!]
\center
\includegraphics[scale=0.5]{inf_fcast}
\end{figure}

\end{frame}

%------------------------------------------------------------------------------------------------------------------------
%                    XX
%------------------------------------------------------------------------------------------------------------------------
\begin{frame}

\frametitle{Išvados}

\begin{itemize}
\item{Suformavome Lietuvos infliacijos modelį, remiantis vartotojų kainų indeksu;}
\item{Tikslių išvadų infliacijos elgesiui būsimajam laikotarpiui pateigti negalime;}
\item{Netinkamumo priežastis: mūsų modelis ignoruoja struktūrinį infliacijos lūžį, kuris buvo Lietuvos ekonominės krizės metu, bei atsižvelgia ne į visus kointegracijos sąryšius;}
\item{Pagilinus ekonometrines žinias bei patirtį, geriausia būtų įvertinti vektorinį modelį, kuris neignoruotų struktūrinių lūžių, bei kointegracijos sąryšių.}
\end{itemize}

\end{frame}
%------------------------------------------------------------------------------------------------------------------------
\end{document}
